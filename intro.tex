\chapter*{Introduction}
\addcontentsline{toc}{chapter}{Introduction}


Biodiversity is a striking feature of our world. Diverse ecosystems
provide humanity with irreplacable services \citep{daily1997} and yet
predicting biodiversity's dynamics---a key goal in order to better
manage and preserve it---continues to ellude ecologists and
evolutionary biologists \citep{mcgill2010, gould1985}. Over the past
550 million years biodiversity has fluctuated between periods of rapid
diversification, such as the Cambrian explosion, and devastating
extinctions, such as the end Permian extinction \citep{sepkoski1992,
  alroy08}. In the Modern, where we can directly sample living
assemgladges of species, surprising regularities are found in the
distributions of population sizes, resource uses and trophic
interactions across species \citep{brown1995, hubbell2001, nekola2007,
  harte2011, Jordano2003, nuismer2013, berlow2009}. Thus to predict
diodiversity we must embrace both the extreme variability and
contingency of its history as well as the persistent emergent
similarities in macroscopic patterns across systems.

Statistical mechanics provides a powerful framework for perdicting
macroscopic generalities in systems composed of many constituient,
potentially complex, parts \citep[e.g.][]{albert2002, sethna2006}.  In
this thesis I use the predictive power of statistical mechanics to
understand the origin of pervasive patterns in biodiversity, such as
the distribution of its fluctuations through time and the distribution
of trophic interactions across species.  I combine this approach with
an explicit consideration of the evolutionary procees leading to these
biodiversity patterns, seeking to understand how biological evolution
drives biological systems away from the steady state idealizations
predicted by statistical mechanics.  Thus I show using examples from
the fossil record and rapidly evolving island biotas that general
patterns in biodiversity emerge from a combination of the statistical
mechanics of large systems and the unique non-equilibrium dynamics
imparted to biological systems by their evolutionary history.

Biodiversity theory seeks to predict observed assembladges of species
\citep{hubbell2001}. Historically, this pursuit has been dominated by
``bottom-up'' theories that attempt to account for all possible
mechanistic interactions between species and their environments
\citep{hubbell2001, haegeman2008, harte2011}, which invariable depend
on the specifics of the systems for which they are developed and
cannot be generally applied across the breadth of biodiversity.  The
pursuit of simple universal theory was ignited by the work of
\citet{macWilson} whose quantitative description of island assembdaged
as a dynamic equilibrium of colonization and extinction paved the way
for more nuanced theories such as the unified neutral theory of
biodiversity \citep{hubbell2001}, predicting the species abundance
distribution of any given local community based on the stationary
distribution of a birth-death-immigration-speciation process.

Surprisingly, the patterns these theories seek to predict do not
appear to be unique to biology \citep{nekola2007}, implying that the
mechanisms leading to these outcomes should not be uniquely
biological.  Indeed, there is historical precidence in ecology to
consider macroscopic properties of assembladges, such as the species
abundance distribuiton, as the outcome of general statistical laws
\citep{fisher1943, preston1950, preston1962a, preston1962b}. This
``non-mechanistic'' line of thinking has culminated in purely
statistical theories of biodiversity, such as the maximum entorpy
theory of ecology \citep{harte2011}. Such theories paralelle the
coarse-graining of statistical mechanics, where the details of how
micro-states interact (e.g. particles in the case of theoretical
physics or populations in the case of biology) are averaged over in
order to predict the behavior of the system as a whole.

The application of coarse-graining in ecology has met with
considerable emperical success \citep[e.g.][]{banavar2007, pueyo2007,
  dewar2008, harte2011}, yet biology differs fundamentally from
statistical physics because of the non-Markovian memory imparted to
species and populations due to common inheretance of genomes and
niche-constructed environments \citep{odling2003} that are the result
of natural selection.  If biological systems are in steady state they
have the chance to overcome the effect of evolutionary contingency,
but far from equilibrium the importance of evolutionary history should
be clearly seen in deviations between observed biodiversity patterns
and statistical idealizations thereof.

Here I use two complementary approaches to understand the importance
of evolutionary history through the lense of statistical mechanics.
In the first I develop a dynamic non-equilibrium theory of changes in
biodivesity by extending a branch of non-equilibrum statistical
physics, known as super-statistics \citep{beck2003, beck2004}, to
demonstrate that the complex dyanimic of diversity fluctuations
through time arrises from the same non-equilibrium processes that can
describe complexity in large physical \citep{beck2004} and social
systems \citep{fuentes2009}.  In the second approach I test a static
equilibrial theory of biodiveristy \citep[the maximum entropy theory
of ecology][]{harte2011} across rapidly evoloving arthropod
communities of different ages in the Hawaiian archipelago. This work
combines population genetics with the geologic chronosequence to
understand how the rate and mode (e.g. through adaptive
differenciation) of evolutionary community assembly drives deviations
from statistical steady sate. To do so I developed an open source
\texttt{R} package implementing an approach to theory development in
ecology based on the principle of maximum information entropy. This
framework seeks to predict distributions of interest, such as
abundance distributions, without invoking any specific mechanistic
assumptions but instead by finding the solution that an ideal system
would reach in equilibrium.  Thus deviations from this theory can
reveal the nature of unique mechanisms behind the observed structure
of biodiversity.


\printbibliography[heading=subbibliography]

