\chapter{Community assembly on isolated islands: macroecology meets
evolution}

\section{Introduction}

Current biodiversity is a product of speciation, extinction and
dispersal, contingent on the ecological interactions of organisms with
their biotic and abiotic environment. The evolutionary history leading
to the assembly of any given ecological community must in some way
shape current ecological assemblages. However, because the processes
of evolution and ecology occur on different temporal and spatial
scales, disentangling the relative influence of local ecological
mechanisms from historical evolutionary processes on patterns of
community structure remains a central challenge \citep{ricklefs2004}.

The evolutionary processes of speciation and extinction are
classically viewed as constraints on regional species pools, occurring
in a manner largely removed from local ecology \citep{hubbell2001,
  cavenderbares2009, wiens2011}. Conversely, ecological mechanisms
tend to be viewed as packing standing diversity into local communities
through consumption, competition, facilitation and, more recently,
neutral ecological drift \citep{hubbell2001, tilman2004,
  bascompte2007, borer2014}. While recent theoretical advances have
provided greater insight into ecological drift \citep{hubbell2001,
  rosindell2011ecolLett}, niche partitioning \citep{tilman2004},
competition, predation \citep{borer2014} and species interaction
networks \citep{williams2000, brose2006}, these insights typically do
not contain realistic evolutionary assumptions
\citep{ricklefs2006neutral} or ignore them entirely.

Insights into the genetic, biogeographic and selective mechanisms
leading to diversification have also emerged based on inference from
current patterns of species, genetic or phylogenetic diversity
\citep[e.g.][]{wiens2011, jetz2012}. However, it is not possible to
use current static patterns to infer the temporal dynamics of either
the evolutionary mechanisms or their ecological consequences, nor can
we understand what constitutes meaningful change in a system without a
baseline for comparison. Here we show how testing idealized ecological
theories---such as the unified neutral theory \citep{hubbell2001} or
the maximum entropy theory of ecology \citep{harte2011}---on
archipelagos composed of islands formed in a discrete geological
sequence can help identify the shifting balance and feedback between
fast-acting, local ‘ecological’ mechanisms, and longterm, large-scale
evolutionary processes in determining ecological community
structure. Islands having different ages of formation, along with
discrete volcanoes within islands, provide the opportunity to study
diversification of species and the assembly of communities in
different stages. Ecological theory provides an idealized ‘null’
baseline against which to compare observed patterns.


\subsection{Hotspot oceanic archipelagos as model systems}

Hotspot oceanic islands are opportune model systems for studying the
interplay of local ecological mechanisms and the evolutionary drivers
of biodiversity patterns. Due to their sequential formation as the
tectonic plate moves over a volcanic hotspot, such island systems
offer a range of spatial and temporal scales over which to analyze the
outcomes of ecological and evolutionary processes
\citep{warren2015}. While many archipelagos around the world share
these biotic and geological properties, the Hawaiian archipelago
provides a particularly useful system for study because its linear
geological chronology \citep{price2002}, ecosystem developmental
trajectories \citep{vitousek2004} and phylogeographic patterns of
biodiversity are each well characterized \citep{wagner1995}. Moreover,
studies of species diversity across the islands have revealed patterns
that are non-uniform across the island chronosequence with marked
differences among lineages \citep[e.g.][]{gruner2007, gillespie2009}
that can be used to test for biologically meaningful differences among
lineages that might drive their disparate diversification patterns.


\subsection{Development of genetic structure}

High levels of dispersal and associated gene flow among localities
limit the extent to which populations can diverge
genetically. However, when gene flow is low, distinct populations in
different localities are free to diverge through local selective
pressures and drift, which can lead to diversification
\citep{slatkin1987} Thus, the magnitude of genetic connectivity among
populations provides a measure of the relative importance of
dispersal-driven assembly (dictated by processes removed from the
local setting) in contrast to assembly by local (\textit{in situ})
diversification in determining community composition. Using the
chronosequence of the Hawaiian archipelago, we can analyze populations
from multiple sets of taxa across trophic guilds occurring in
geological contexts from young to old. We predict that
dispersal-driven (ecological) processes will dominate in community
assembly in young habitats, with the importance of in situ
(evolutionary) processes increasing with habitat age. If evolutionary
processes are not important, we predict that communities should reach
a statistical steady state through ecological processes alone
\citep{harte2011}. If, as we expect, evolutionary processes become
increasingly important in community assembly over time, we would
expect to find associated deviations from an ecological null model of
community assembly, provided by idealized ecological
theory. Differences in population structure among taxa or trophic
groups could indicate whether sufficient time has passed along the
chronosequence for the group of interest to experience significant
evolutionary pressures.


\subsection{Macroecological metrics and idealized ecological theory}

By their nature, unified theories of biodiversity
\citep[e.g.][]{hubbell2001, harte2011} provide a simplified view of
ecology, but deviations from theory can provide insights into which
particular ecological patterns require additional biological
mechanisms for their explanation \citep{harte2011}. The maximum
entropy theory of ecology \citep[METE;][]{harte2011} in particular
provides predictions of species abundance distributions, species–area
relationships and metabolic rate and network linkage distributions for
idealized ecological communities in which the behavior of a system is
governed by a simple set of state variables. The principle of maximum
information entropy (MaxEnt), from which the METE is derived, is an
established inference procedure that has yielded accurate predictions
of diverse patterns in fields as varied as thermodynamics (Jaynes,
1957), economics \citep{golan1996}, forensics \citep{roussev2010},
imaging technologies \citep{gull1986} and, more recently, ecology
\citep[e.g.][]{phillips2006, dewar2008, harte2011}. MaxEnt works by
seeking the least-biased prediction of a distribution of interest
(e.g. the distribution molecular velocities in the case of
thermodynamics or of species abundances in the case of ecology) while
constraining that prediction to be consistent with state variables
describing the macroscopic attributes of the system (e.g. temperature
or the total number of species and individuals). These are the most
ignorant possible predictions about the system. Thus, studying the
unique ecological conditions and evolutionary histories of real-world
systems that deviate from the conditions predicted from maximizing
information entropy can provide insights into the processes driving
ecological systems away from the statistical steady state
\citep{harte2011}.

Ecological networks are complex systems forming hierarchical
structures to which the principle of MaxEnt has recently been applied
\citep{williams2010, harte2011} and are a prime study focus because
networks of interacting species embody both the ecology of trophic
links and evolutionary processes such as co-evolution
\citep{bascompte2007, donatti2011, nuismer2013, thompson2013}. Thus
they present an opportune starting place to study ecological and
evolutionary feedbacks. The distribution of linkages in ecological
networks can test whether plant–animal interaction networks assemble
neutrally or through deterministic processes such as co-evolution of
traits involved in foraging \citep{vazquez2005}. Analysis of other
network metrics such as modularity (the degree to which species
interact in semi-autonomous modules) and nestedness (the degree of
asymmetry in interaction between specialists and generalists) can
further illuminate the underlying eco-evolutionary processes driving
patterns of species interactions \citep{bascompte2007, donatti2011,
nuismer2013}. In nested networks, species with fewer interactions
(i.e. more specialized species) will interact with a subset of the
species with which generalists interact. In this way interaction
nestedness is mathematically equivalent to island nestedness (in which
islands that are less species rich are subsets of islands that are
more species rich). However, we only consider network nestedness here.

To gain insights into community assembly as it happens, we propose an
integrative framework that harnesses advances in both evolutionary and
ecological theory, placed in the context of age-structured
archipelagos. Mechanistically simplified ecological theories such as
the METE \citep{harte2011} can be used as powerful null models;
deviations from theoretical expectations can flag biological phenomena
that warrant further study. Here we demonstrate how community-level
data from age-structured island systems, combined with population
genetic and phylogenetic data, can test the extent to which the
evolutionary histories behind such communities drive their deviation
from theoretical expectations. We provide an initial test of this
concept using a synthesis of published data on arthropod lineages in
the Hawaiian islands. We provide metrics of ecological and
evolutionary dynamics across communities from settings that range in
geological age from 500 years to 5 Ma. We estimate taxon-specific
timelines for the development of population genetic structure for both
herbivores and predators and couple these results with macroecological
measures of community structure, using predictions from statistical
steady-state and ecological network theory to provide insights into
changes in community structure over the extended timeframe provided by
the island chronosequence.


\section{Methods}

\subsection{Dispersal-driven processes to in situ differentiation
across the island chronosequence}

To evaluate the balance between regional immigration and the potential
for local differentiation, we measured how molecular variation is
partitioned among populations within species across locations of known
substrate age on the islands of Hawaii and Maui (Fig. 1). We compiled
published [DNA sequences, amplified fragment length polymorphism
(AFLPs) and allozymes] and new data sets for a diversity of native
Hawaiian arthropod groups that represent a spectrum of trophic levels
(Table1). New sequences were included for sap-feeding Hemiptera group
\textit{Nesosydne} planthoppers [COI; data generated following the
protocols in \cite{goodman2012}; GenBank accession numbers:
KT023113–KT023179] and \textit{Trioza psyllids} [COI, cytB; data
generated following protocols in \cite{percy2003}; GenBank accession
numbers: KR108061–KR108144]. Samples were from the focal sites
described below for the ecological analysis, as well as from other
locations across Hawaii and Maui. These data were used to provide an
estimate of how arthropod populations have accumulated genetic
population structure within the focal sites of different geological
age.

We used analysis of molecular variance (AMOVA) to examine how genetic
variation is partitioned at two scales of population structure: among
sites within volcanoes and among volcanoes on both the island of
Hawaii and the islands of the Maui Nui complex (Maui, Molokai,
Lanai). All analyses of allozyme and DNA sequence data were performed
in \texttt{Arlequin} v.3.5 \citep{arlequin} using the AMOVA procedure
to compute $F_{ST}$, a measure of genetic variance, or, where
possible, $\Phi_{ST}$, an $F_{ST}$ analogue that incorporates genetic
sequence information. The \textit{Laupala} AFLP data were analyzed
using \texttt{tfpga} v.1.3 (Miller, 1997), using the same hierarchical
approach of comparing within and among volcanoes as described
above. To provide a temporal framework for the population
differentiation analysis we assembled divergence-dating information
from the literature for as many of the taxa as possible.

To explicitly test the association between landscape age and the
potential for in situ genetic divergence we analyzed how within-site
$F_{ST}$ varies with the geological age of volcanoes on the islands of
Hawaii and Maui Nui. For each volcano we calculated $F_{ST}$ or
$\Phi_{ST}$ \citep{arlequin} for each taxon among sites within
volcanoes. This analysis assumes that volcano age parallels habitat
age, allowing more or less time for the presence of the populations.


\subsection{Ecological metrics across the island chronosequence}

To investigate how ecological patterns change as communities age, we
selected four focal sites across the chronosequence and island ages
(two on the island of Hawaii, one on Maui and one on Kauai; Fig. 1) of
approximately 12 km$^2$ (each was defined as a point with a 2 km
radius buffer). Focal sites were selected to have similar forest
composition (dominated by \textit{Metrosideros polymorpha};
Myrtaceae), elevation (1100--1400 m) and rainfall (mean annual
precipitation 2000--3000 mm). We then constructed bipartite
interaction networks between native herbivorous Hemiptera species and
native plants at each of the study sites. Bipartite networks describe
the topology of ecological interactions between two guilds of
organisms (e.g. herbivores and their plant hosts). Quantitative
information on the relative importance of interaction links can be
incorporated into network analyses (Vazquez et al., 2009). However,
currently available data are restricted to binary networks: those that
describe the potential for interaction between any two species but not
the relative frequency of that interaction to each species.

We compiled species lists of all native herbivorous Hemiptera for each
focal site from published species accounts (see Table S1 in the
Supporting Information for a full list). Species accounts and other
published sources were used to determine the presence, probable
presence, or probable absence of each species at each of our four
focal sites. A documented presence was defined as a known specimen
collected at the focal site; a probable presence was defined as a
species whose abiotic tolerances and known geographic range overlap
with a focal site but no known specimen exists confirming its
presence. Probable absence was assumed when the criteria for presence
or for probable presence are not met. Two sets of species lists for
each focal site were compiled: a conservative data set composed of
only documented presence occurrences and a less conservative data set
that also included probable presences.

Host plants for each species of Hemiptera were determined from
published species accounts. Data on host plant use at each specific
site were not available so we assumed that if a known host plant were
present at a site it would eventually be used. Host plant occurrence
in the focal sites was determined using distribution models for 1158
species of Hawaiian plants \citep{price2012}. Each focal site was
spatially joined in a geographic information system with all
coincident plant distribution models that fell within its
boundaries. Two sets of resulting focal site-specific networks were
constructed: one using the conservative data set of Hemiptera species
presences and the other using the less conservative data set.

We hypothesized that potentially complex evolutionary feedbacks
contributing to community assembly should result in departures from
the predicted ecological statistical steady state. We used the METE
\citep{williams2010, harte2011} to compute the statistical steady
state for the distribution of the number of host plants used by each
Hemiptera species (hereafter referred to as degree distribution). To
evaluate how well the METE predicts the data we simulated
METE-conforming communities having the same number of species and
links as observed. We then calculated the log-likelihood of each
simulated data set and compared the resultant distribution of
log-likelihoods under the hypothesis that the METE is true with the
observed log-likelihood. This comparison is identical in approach to a
z-score test using a Monte Carlo simulation to estimate the sampling
distribution of log-likelihoods. \texttt{R} scripts
\citep[v.3.1.1;][]{RCore} used for METE estimation and Monte Carlo
methods are available in Appendix 1.  To investigate how speciation
may in part drive network patterns and deviations from those predicted
by idealized ecological theory, we analyzed the number of links
assigned to each Hemiptera species (the degree distribution)
separately for single-island endemics (those species found on only one
island and thus probably derived from in situ diversification) versus
multi-island endemics (those species found on multiple
islands). Although multiple processes can lead to a species being a
single-island endemic \citep{whittaker2008}, such taxa provide a proxy
for how much speciation occurs within islands. To compare species
degree distributions between single-island endemics and multi-island
endemics across sites of different ages we conducted a generalized
linear model with binomial error, treating site identity as a
categorical predictor. Binomial errors effectively account for network
size due to the bounded support of the binomial distribution.

To understand how other network properties change with age of the
ecosystem substrate, we calculated two widely used descriptive network
metrics across sites–--nestedness and modularity. Nestedness describes
the degree of asymmetry of species interactions connecting specialists
and generalists \citep{bascompte2007, ulrich2009}. We calculated
nestedness using the NODF metric \citep{nodf} as implemented in the
\texttt{R} package \texttt{vegan} \citep{vegan} and modularity using a
variety of algorithms implemented in the \texttt{R} package
\texttt{igraph} \citep{igraph}. These metrics are not directly
comparable across networks of different size and connectance
\citep{ulrich2009}, so for each metric in each network we calculate
z-scores using a null model that randomizes network structure while
maintaining certain aggregate network properties (Ulrich et al.,
2009). These z-scores are calculated as the difference between the
observed network metric minus the mean of the null model divided by
the null model standard deviation, or $(x_{obs} − \bar{x}_{sim}) /
SD_{sim}$. Because z-scores can be highly sensitive to the choice of
null model \citep{ulrich2009} we implemented both a probabilistic null
model \citep{bascompte2007} and a null model that strictly constrains
the degree distributions of plants and herbivores
\citep{ulrich2009}. The probabilistic null uses the frequency of
interactions as the probability that a randomized link gets assigned
to that cell in the interaction matrix \citep{bascompte2007}; thus the
probabilistic null constrains row and column sums in probability but
not absolutely.


\section{Results}

\subsection{Dispersal-driven processes to in situ differentiation
across the island chronosequence}

The AMOVA revealed significant genetic population structure from the
smallest to the largest spatial scales examined, all within a very
recent timeframe. For mitochondrial loci, statistically significant
molecular variation partitioned among sites within volcanoes ranged
from 0.037 to 0.92 and among volcanoes from 0 to 0.30. Corresponding
variation at multilocus nuclear loci among sites within volcanoes
ranged from 0.21 to 0.58 and among volcanoes from 0.04 to 0.34. Taxa
in the lower trophic levels (herbivorous sap-feeding Hemiptera:
planthoppers and psyllids) had as much or more molecular variation
partitioned at the among-site, within-volcano level than the
among-volcano level, while the predatory spiders were less structured
at localities within volcanoes compared with among them (Table 1). The
analysis of genetic population structure across the chronosequence of
localities revealed a similar pattern. The herbivores show high
genetic population structure among localities even on young volcanoes
(Fig. 2). By contrast, predatory spiders exhibited little genetic
population structure within sites on the same volcano; this was higher
among volcanoes, with values increasing with age across the
chronosequence.

The observed levels of genetic divergence have evolved rapidly in many
cases. For example, for species from the island of Hawaii for which
phylogenetic data provide divergence times, estimates of dates of
species divergence range from 0.5--4 Ma, with additional
within-species genetic divergence having developed subsequently (Table
1). That some of these estimates are older than the known age of the
‘Big Island’ suggests that genetic divergence pre-dates their
colonization to Hawaii, or alternatively that estimates include
sampling error. For the one species where population genetic data were
used to estimate divergence times between populations, herbivorous
\textit{Nesosydne} planthoppers, it was determined that populations
diverged as little as 2600 years ago \citep[][Table 1]{goodman2012}.


\subsection{Ecological metrics across the island chronosequence}

The degree distribution of Hemiptera species varied across the
chronosequence with both the youngest and oldest sites deviating most
from the statistical steady-state maximum entropy predictions
(Fig. 3). In the intermediate-aged site of Kohala, deviations are not
significantly different from the predictions of maximum entropy.  The
generalized linear model revealed significant differences between the
degree distributions of single-island endemics (species whose
distributions are restricted to only one island) versus archipelagic
endemics that are found across multiple islands
(Fig. 3). Single-island endemics show significantly lower degree
distributions overall (i.e. more specialization) compared with more
generalist species found across multiple islands. Furthermore,
single-island endemics use more host plant species on the
intermediate-aged Maui site. The slightly younger Kohala shows
increased generalization for both single-island endemics and
archipelago endemics. However, when considering the degree
distribution defined by trophic links to plant genera instead of plant
species, the pattern of increased generalization holds for Kohala, but
endemics on Maui no longer show a difference in their degree
distributions from other island endemics. This change in pattern
suggests that increased generality of Maui endemics may be driven by
increased plant species diversity within genera on that island.

Network nestedness decreased with habitat age while modularity
increased (Fig. 4). This trend was recovered in networks constructed
from both more and less stringent geographic criteria
(Fig. S3). Choice of null model changed the magnitude of modularity
and the sign of nestedness z-scores; however, the relative pattern of
decreasing nestedness and increasing modularity remained across the
different null models used to standardize network metrics
(Fig. S2). The patterns were also robust to sampling intensity, as
demonstrated by a rarefaction analysis (Fig. S4).


\section{Discussion}

\subsection{Development of genetic population structure at different
trophic levels}

The analysis of available genetic data presented here indicates that
divergence is occurring within the islands at small spatial scales and
over short time periods (Table 1, Fig. 2). Furthermore, the scale of
population structure varies with trophic position, with structure
developing in sap-feeding herbivore lineages at smaller scales (and
hence shorter timeframes in the context of the chronosequence)
compared with detritivorous crickets and predatory spiders (Table 1,
Fig. 2). Structure within species may allow populations to take
independent evolutionary trajectories, especially when aided by other
evolutionary processes acting differentially across species geographic
ranges. A variety of factors have been associated with the genetic
divergence of populations and species in the lineages described here,
including combinations of genetic drift associated with geographic
isolation \citep{percy2003, mendelson2005, ogrady2011, goodman2012},
adaptation associated with competition, predation and mutualism
\citep{gillespie2004, roderick2008, brewer2015} and sexual signaling
\citep{mendelson2005, percy2006, magnacca2008, goodman2015}.

The \textit{Nesosydne} planthoppers provide evidence that some period
of geographic isolation preceded the divergence of sexual signals
\citep{goodman2012, goodman2015}. Shifts in plant host use are also
associated with diversification in this group \citep{roderick2008}. In
a phylogenetic study of a radiation of sap-feeding
\textit{Nesophrosyne} (Cicadellidae) leafhoppers, species divergence
was associated with host plant specialization between 1 and 5 Ma, but
only with geography on the younger island \citep{bennett2013}. Our
network analysis shows that specialization and modularity are more
pronounced on Maui than on Hawaii (Figs 3 and 4), consistent with the
phylogenetic results from \textit{Nesophrosyne}. Available dating
analyses of other arthropod taxa indicate that population genetic
structure can develop in much less than 1 Myr (Table 1), and suggest
that landscape fragmentation processes (e.g. lava flows) may dominate
the earliest stages of diversification across taxa in the Hawaiian
islands. Other taxa at low trophic levels, such as the herbivorous
\textit{Trioza psyllids}, detritivorous \textit{Laupala} crickets and
fungivorous Drosophila, show similar signals of geographic isolation
combined with ecological and sexual processes driving genetic
divergence and diversification across sites as young as those on
Hawaii \citep{percy2003, mendelson2005, percy2006, magnacca2008,
  ogrady2011}. By contrast, spiders, which are predatory, develop
genetic discontinuities at larger spatial and temporal scales with a
strong signature of increasing structure with age of the
chronosequence \citep[][Table 1]{roderick2012}. Further work is needed
to assess the generality of this pattern of slower genetic
differentiation in predators compared with herbivores.


\subsection{Macroecological metrics: network structure and steady
state}

Across the Hawaiian archipelago, nestedness appears to decrease
generally with site age, and is highest on the geologically youngest
volcano, Kilauea. High nestedness on Kilauea may arise with high
immigration of new species with high probabilities to eat or be eaten
by the generalist species already present at the site
\citep{bascompte2007}. However, despite high nestedness on Kilauea,
and thus the potential for neutral colonization-driven assembly, this
site did not conform to the statistical steady-state predication of
the METE. The observed deviations from the METE at Kilauea appear to
be largely driven by a surplus of singleton links (Fig. 3), which may
reflect a state of ‘incomplete’ assembly, possibly by lower species
richness of the plant and herbivore biotas. Conversely, at Kohala, at
intermediate age (150 ka), observations were not significantly
different from the METE predictions. We posit that the reason why
theoretical predictions fit Kohala so well is that the site has had
sufficient time to undergo ecological succession and thus arrive at a
statistical steady state, but is still too young to be affected by
ecological specialization and rapid in situ diversification associated
with host plants on older islands.

Interestingly, the communities on the older Maui and Kauai sites show
strong deviations from the METE expectations (Fig. 4). The METE is
agnostic about which mechanisms determine the values of the state
variables that lead to its macroecological predictions (Harte,
2011). It does not account for the evolutionary history of biological
systems. Thus, one possible explanation for the strong deviations from
the METE expectations, compared with observations at our
intermediate-aged site (Kohala), is that while the ages of Maui and
Kauai are sufficient for evolutionary assembly driven by
specialization and diversification on host plants, the older age of
these islands may have led to range contractions and possibly
extinction of plant species on the oldest island of Kauai (Whittaker
et al., 2008).

Our results show decreased nestedness and increased modularity on Maui
and Kauai. Co-evolution between interacting species should lead to
greater modularity \citep{donatti2011, nuismer2013}. However, the
influence of certain network properties, such as nestedness, on
stability is still unknown, and so theoretical predictions of how
network properties should change over evolutionary time, generally,
are lacking. Theoretical and empirical studies have suggested that
nestedness may or may not promote stability \citep{allesina2012,
suweis2014}. Furthermore, almost all studies of food webs have focused
primarily on single or short ecological time spans of network
development that do not span as much evolutionary time as is included
here \citep[e.g.][]{albrecht2010}. Food webs are dynamic emergent
entities, with broad topological characteristics that may change
dramatically over time \citep[e.g.][]{yeakel2013}. To our knowledge,
our study represents the first to evaluate network topology over
larger temporal scales, and we argue that age-structured landscapes
such as the Hawaiian archipelago are promising for resolving
long-standing debates on the causes and consequences of network
properties such as nestedness.

We found that single-island endemics were always more specialized than
multiple-island endemics. Although dietary breadth has been positively
associated with geographic range size \citep{lewinsohn2005}, the
direction of causality is unclear \citep{slatyer2013}: while dietary
breadth may allow some species to colonize other islands, it may also
be driven by adaptation to exploit locally abundant hosts across a
large range. Nevertheless, both scenarios are consistent with the
hypothesis that in situ formation of single-island endemics may be the
product of co-evolution and specialization. At the Kohala site, which
showed the best fit to maximum entropy theory, single-island endemic
and multiple-island endemic species alike showed increased
generalization (i.e. a higher degree, or more links; Fig 3), while at
the youngest site of Kilauea, specialist single-island endemics may be
limited by low plant diversity and thus appear more specialized
(Fig. 3). Conversely at the oldest site on Kauai, where plant
diversity is high \citep{kitayama1995}, single-island endemics are
again associated with decreased degree and thus genuine specialization
(Fig. 3). On Maui, single-island endemics show statistically
significant increases in generalization, but this pattern disappears
when analyzing the data at the resolution of plant genera, thus
suggesting that Hemiptera species endemic to Maui may benefit from the
diversification of plant species within genera.


\subsection{Future research}

The data and analyses presented here describing insect and plant
communities across a chronosequence of habitats in Hawaii generate
testable hypotheses concerning the relative importance of ecological
and evolutionary processes in community assembly. Our work to date
suggests the overarching hypothesis that ecological processes dominate
community assembly in younger environments, with evolutionary
processes becoming increasingly important as communities age. We can
also make predictions about the sequence of community assembly based
on proposed mechanisms.

In younger communities we predict characteristics of ecological
assembly, with species resembling random samples through immigration
from regional source pools. Thus, metrics describing these communities
will approach expectations of an ecological statistical steady
state. An exception will be communities that are still undergoing the
initial stages of primary succession, which will change rapidly
through time and represent nonrandom samples of source pools. We also
predict that these communities will exhibit a nested network
structure, assuming new species will eat or be eaten by the generalist
species already present in the community, as suggested by previous
work on nestedness (bascompte2007) and by our finding that widespread
species tend to be generalists (Fig. 4).  Following the same logic, in
older communities we expect to see characteristics of evolutionary
assembly, dominated by processes such as adaptive exploration of niche
space, giving way to speciation. Thus, we predict increasing
specialization and modularity with time \citep{bascompte2007,
donatti2011, nuismer2013} as reflected by age across the
chronosequence.


\subsubsection{Ecological data: assembly of species into communities}

In order to build a more rigorous understanding of the assembly
process in both younger and older communities, fine-grained sampling
of all macroscopic arthropod taxa is needed from a large number of
sites across the island chronosequence. This will allow an assessment
of changes in overall species composition and diversity across all
players in the time-calibrated landscape \citep{gruner2007}. Such data
will allow us to test entire arthropod communities for deviations from
METE predictions of statistical steady state \citep{harte2011} across
substrates of different ages. For example, predators, whose
assemblages are likely to be more dominated by immigration and
ecological assembly (Fig. 2), may never show strong deviations from
METE predictions, whereas herbivores could show increasing deviation
with age in agreement with the network results of this paper (Fig. 3).


\subsubsection{Evolutionary data: diversification within species} The
current study demonstrates that taxa from different trophic guilds
differ in the scale at which differentiation occurs and highlights the
importance of fragmentation of the landscape in facilitating
differentiation. Future work will be aimed at gathering data for
additional focal taxa within this system, spanning different trophic
levels. We will use these data to understand taxonomic and functional
differences in the rate of differentiation, to assess the roles of
genetic fusion and fission and the spatial scale over which they are
important in fostering diversification \citep{gillespie2014}, and to
detail the relative rates of speciation and extinction across the
island chronosequence.


\section{Conclusions}

We have shown how a chronosequence can be used to understand
biodiversity dynamics across an ecological–evolutionary
continuum. Focusing on entire communities of arthropods in the
Hawaiian islands allows us to incorporate predictions from idealized
ecological theories to understand eco-evolutionary feedbacks and
generate predictions about how entire communities develop over an
extended time. Such an approach may prove fruitful for investigating
the separate and interactive roles of ecological and evolutionary
drivers of community assembly using age-structured systems as a
simplified natural experiment, as exemplified by oceanic archipelagos.

We have demonstrated how taxa in the lower trophic levels developed
genetic structure even in the youngest habitats of the observed
chronosequence and at smaller spatial scales (Table 1, Fig. 2). Thus,
lower trophic levels are affected by in situ processes of
diversification very early in the chronosequence, compared with higher
trophic levels, though in situ processes become more important over
time in the latter. Network nestedness decreased while modularity
increased with age (Fig. 4), again indicating a possible shift from
assembly driven by ex situ immigration early on to one based on in
situ diversification, such as in co-diversification of insect
herbivores with host plants \citep{bascompte2007, donatti2011}. That
single-island endemics (probably the product of in situ
diversification) show more specialization at older sites than more
broadly distributed species (those taxa more likely to be initial
colonists; Fig. 3) also supports this hypothesis.

This study provides a framework for using chronologically arranged
oceanic island systems to examine the interplay between evolutionary
and ecological processes in shaping biodiversity. Our initial results
provide a clear hypothesis that ecological processes dominate
community assembly in younger environments, with evolutionary
processes becoming more important as communities age. We demonstrate
how this approach can provide insights into the development of
communities over ecological–evolutionary time, and the dynamic
feedbacks involved in assembly.


\section*{Acknowledgements}

We are indebted to many scientists and land managers in Hawaii who
have provided access to the lands: Pat Bily (The Nature Conservancy of
Hawaii), Melissa Dean, Christian Giardina, and Tabetha Block (Hawaii
Experimental Tropical Forests), Betsy Gagne (Natural Area Reserve
System), Lisa Hadway and Joey Mello (Department of Forestry and
Wildlife Hilo), Cynthia King and Charmian Dang (Department of Land and
Natural Resources) and Rhonda Loh (Hawaii Volcanoes National Park). We
thank Robert Ricklefs, Lauren Ponisio and two anonymous referees for
thoughtful commentary. We are very grateful to Guida Santos, Richard
Field and Robert Ricklefs for inviting us to contribute to this
Special Issue. The research was supported by the National Science
Foundation DEB 1241253.

% \bibliographystyle{thesis}
% \bibliography{thesis}

\clearpage

\section*{Figures}

\begin{figure}[!hp]
  \centering
  \includegraphics[scale=0.35]{figs/fig_map.pdf}
  \caption[Map of substrate age]{Map of substrate age (millions of
years, My) of the islands of Kauai, Maui and Hawaii. Colours
correspond to substrate age from young (light) to old (dark). Focal
sites are shown as black circles (on Hawaii, Kohala is in the north,
Kilauea in the south) while sampling sites for genetic data are
represented by grey circles..}
  \label{fig:map}
\end{figure}

\begin{figure}[!hp] 
  \centering
  \includegraphics[scale=0.35]{figs/fig_volcanoFst.pdf}
  \caption[Population genetic structure]{Population genetic structure
($\Phi_{ST}$ for all taxa except \textit{Laupala} for which we used
$F_{ST}$) among sites within volcanoes with volcano age for insects
and spiders. Calculations were based on mitochondrial DNA only (see
Table 1 for details). The plant-feeding groups, specifically the
sap-feeding Hemiptera, show higher genetic structure among sites on
young volcanoes relative to older volcanoes, whereas detritivores
(crickets), fungivores (\textit{Drosophila}) and in particular
predators (spiders) show little structure on young volcanoes. For
spiders, substantial structure develops only later in the
chronosequence, for example on Maui at approximately 1 Ma. Numbers
refer to different species: 1, \textit{Nesosydne chambersi}; 2,
\textit{Nesosydne raillardiae}; 3, \textit{Nesosydne bridwelli}; 4,
\textit{Trioza} HB; 5, \textit{Trioza} HC; 6, \textit{Drosophila
sproati}; 7, \textit{Laupala cerasina}; 8, \textit{Tetragnatha
anuenue}; 9, \textit{Tetragnatha brevignatha}; 10, \textit{Tetragnatha
quasimodo}; 11, \textit{Theridion grallator}.}
  \label{fig:popGen}
\end{figure}

\begin{figure}[!hp] 
  \centering
  \includegraphics[scale=0.35]{figs/fig_degree.pdf}
  \caption[Patterns in degree distributions across sites]{Patterns in
degree distributions across sites, comparing archipelago-wide endemics
(cosmopolitans) with single-island endemic (Endemics) taxa. The top
panels show that networks deviate most from the predictions of the
maximum entropy theory of ecology on the youngest and oldest
sites. Inset figures show the distribution of simulated mean squared
errors; if the vertical/red line falls within the grey region (95\%
confidence interval) the data are not significantly different from the
predictions of maximum entropy theory. All sites except Kohala deviate
from the predications. The bottom panel shows the number of links for
endemics versus cosmopolitans. Endemics show lower linkage overall,
but significantly increase on the intermediate-aged site Maui
(highlighted with dotted box). Kohala shows increased linkage overall
(highlighted with a solid box).}
  \label{fig:degree}
\end{figure}

\begin{figure}[!hp] 
  \centering
  \includegraphics[scale=0.4]{figs/fig_netMets.pdf}
  \caption[Patterns in degree distributions across sites]{Trends in
    network metric nestedness and modularity through time. Nestedness
    decreases while modularity increases. Error bars represent 95\%
    confidence intervals from a null model simulation.}
\label{fig:netMet}
\end{figure}

\clearpage

\section*{Tables}

\begin{sidewaystable}[!h]
% \begin{table}[!hp] 
  \centering
  \includegraphics[scale=1]{figs/tab_amova.pdf}
  \caption[Results of the analyses of molecular variance]{Results of
the analyses of molecular variance (AMOVA) that partitions molecular
genetic variation among volcanoes and among sites within volcanoes for
arthropod lineages found within the study sites on the island of
Hawaii. Where estimates of divergence through molecular dating are
available for the taxa, they are presented to show the timeframe
within which this genetic structure has developed.}
  \label{tab:genStat}
% \end{table}
\end{sidewaystable}
