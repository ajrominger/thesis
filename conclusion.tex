\chapter*{Conclusion}
\addcontentsline{toc}{chapter}{Conclusion}

how has thesis provided better top-down predictive theory for
biodiversity and better integrated evolution into theory?

use of stat mech to make detailed top-down predictions and to
indentify where other mechanisms needed

application of top down modeling to evolutionary process and testing
of top down theory in evolving landscapes to understand how
evolutionary drives deviation from theory

provided software tools to make theory testing more robust and
accessible

what future steps are motivated by the thesis?



Using two seminal paleontological datasets \cite{sepkoski1992,
  alroy08} I show that fluctuations in marine biodiversity over the
past 550 million years results from the superposition of many
independently fluctuating subsystems whose fluctuations are Gaussian
but give rise to non-Gaussian patterns when combined.  These
independent subsystems correspond to lineages of closely related
animal taxa, implying that diversification within lineages is driven
by random additive interactions with the environment. These findings
thus challenge the idea that changes in origination and extinction
through deep geologic time are the result of complicated evolutionary
interactions among organisms and between organisms and their
environment \cite{bak1993, sole1997, newman1995}. However, I
demonstrate that the evolutionary process responsible for generating
new lineages varies slowly through time, possibly driven by non-random
evolutionary innovations in the physiology and demography of new
lineages. This slow change between lineages produces patterns of
apparent complexity earlier ascribed to unnecessarily complicated
mechanisms. I further show, using permutational null models, that our
findings are not an artifact of how fossils are taxonomically
classified but rather capture true underlying biological processes.







\paragraph{Future work: adaptive landscapes and memory.}
Ecological theory has of late been dominated by neutral models
\citep[e.g.][]{rominger2009, rominger2015GEB}. This approach needs
robust alternative hypotheses because a lack of alternative theories
rooted in classical ecology could be one limitation preventing a more
rigorous competition between deterministic and statistical theories of
biodiversity. My work in the fossil record motivates a new approach to
ecological theory using super-statistics to parsimoniously capture the
non-neutrality of species and relate that non-neutrality to the
non-equilibrium process of diversification on an adaptive landscape. I
am also excited by the prospect of exploring what other principles
from statistical physics could help us illuminate mechanisms in
ecology and evolution. Spin glasses, with their non-exponential memory
and non-ergodic dynamics, could be an ideal starting point.

\paragraph{Future work: metabarcoding and ecological theory.}
Combining test of ecological theory with massive phylogenetic data
could determine what aspects of evolutionary history specifically
shape ecological communities, but massive phylogenetic data on the
scale of large ecological studies is limiting.  A new approach
dubbed ``metabarcoding'' \citep{taberlet2012} harnesses next
generation sequencing technology to produce massive amounts of genetic
data for ecological samples. This approach, however, has known
pitfalls but these could be overcome bioinformatically, a challenge
that could make a huge contribution to the field and one which I am
excited to undertake and have been invited to explore by {\it Trends
  in Ecology and Evolution} \citep{romTREE}.