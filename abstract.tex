\begin{abstract}


\cite{darwin} concluded {\it On the Origin} with his famous remark
about the diversity of a tangled bank of vegetation and the
possibility that it behaves according to some unknown set of universal
laws. After a century and a half, that governing rule set still eludes
us. This persistent scientific puzzle inspires my research and I
approach it with the hypothesis that general patterns in biodiversity
emerge from a combination of the statistical mechanics of large
systems and the unique non-equilibrium dynamics imparted to biological
systems by their evolutionary history. Understanding biodiversity
dynamics comes at a critical moment when human systems are disrupting
those very dynamics. My work will help provide a foundation for how
global change might drive future shifts in ecology and evolution. To
date I have studied macroscopic patterns in the richness, relatedness
and abundance of species using principles from statistical physics and
data science combined with global databases of museum specimens and
exhaustive ecological surveys.  I have collected these data firsthand,
giving me both the insight of a theoretician and the practicality and
intuition of a field biologist. I am excited to enrich my research by
collaborating with scientists who not only study large biological
systems but also social systems, which we must understand equally well
to prevent a biodiversity crisis, and who use principles from
statistical physics to successfully predict generalities in light of
complexity.

what each chapter does

Working with SFI affiliates Pablo Marquet and Miguel Fuentes I
expanded a branch of non-equilibrial statistical mechanics, known as
super statistics, to explain previously unaccounted for wild
fluctuations in the richness of taxa through the Phanerozoic marine
invertebrate fossil record and show how this non-equilibrium is driven
by clades' punctuated exploration of their adaptive landscapes
\citep{romSStat}. To test this theory I developed object-oriented
software that queried online databases of global fossil specimen
records and then calculated Monte Carlo and likelihood-based measures
of model fit.

Our study is the first to demonstrate that complex, previously
unexplained patterns in the sequence of origination and extinction
events in the fossil record are the result of a simple underlying
process emerging from non-equilibrium evolution on an adaptive
landscape \cite{eldredgeGould1972, newman1985adaptive}. Our theory
provides a novel explanation for deep time diversity dynamics invoking
emergence of lineage-level traits as the drivers of complexity via the
same mechanisms by which complexity emerges in large physical
\cite{beck2004} and social systems \cite{fuentes2009}. In the context
of fossil diversity we show how this complexity arises naturally from
the uniquely biological mechanisms of punctuated adaptive radiation
\cite{eldredgeGould1972, newman1985adaptive, hopkins2014} followed by
long durations of niche conservatism \cite{ackerly2003, roy2009range,
  hopkins2014} and thus identify these mechanisms as sufficient and
necessary to produce observed patterns in the fossil record.

Using two seminal fossil datasets \cite{sepkoski1992, alroy08} we show
that fluctuations in marine biodiversity over the past 550 million
years results from the superposition of many independently fluctuating
subsystems whose fluctuations are Gaussian but give rise to
non-Gaussian patterns when combined.  These independent subsystems
correspond to lineages of closely related animal taxa, implying that
diversification within lineages is driven by random additive
interactions with the environment. Our findings thus challenge the
idea that changes in origination and extinction through deep geologic
time are the result of complicated evolutionary interactions among
organisms and between organisms and their environment \cite{bak1993,
  sole1997, newman1995}. However, we demonstrate that the evolutionary
process responsible for generating new lineages varies slowly through
time, possibly driven by non-random evolutionary innovations in the
physiology and demography of new lineages. This slow change between
lineages produces patterns of apparent complexity earlier ascribed to
unnecessarily complicated mechanisms. We further show, using
permutational null models, that our findings are not an artifact of how
fossils are taxonomically classified but rather capture true
underlying biological processes.


Applying static ecological theories in rapidly evolving ecosystems
can highlight what about the evolutionary process drives communities
away from statistical idealizations. Using the chronosequence afforded
by the Hawaiian Islands to capture snapshots of arthropod communities
at different evolutionary ages and stages of ecological assembly, I
have tested static ecological theories of herbivory network structure
and the distributions of abundances and metabolic rates across
species. To do so I developed an open source R package implementing an
approach to theory development in ecology based on the principle of
maximum information entropy. This framework seeks to predict
distributions of interest, such as abundance distributions, without
invoking any specific mechanistic assumptions but instead by finding
the solution that an ideal system would reach in equilibrium.  Thus
deviations from this theory can reveal the nature of unique mechanisms
behind the observed structure of biodiversity.  My studies of network
linkage, abundance and metabolic distributions indicate that rapid
assembly from immigration and speciation in young ecosystems and
extinction in old ecosystems could drive observed patterns.

meteR

\end{abstract}
