\begin{abstract}
  Since at least the time of Darwin biologists have searched for a
  simple set of universal governing mechanisms that dictate the
  dynamics of biodiversity. While much progress has been made in
  understanding system-specific processes and in documenting the
  context-dependent roles of such mechanisms as competition and
  facilitation, we still lack a universal governing rule set.  The
  goal of understanding and predicting biodiversity dynamics comes at
  a critical moment when human systems are disrupting those very
  dynamics. In this thesis I approach this long-standing problem with
  the hypothesis that general patterns in biodiversity emerge from a
  combination of the statistical mechanics of large systems and the
  unique non-equilibrium dynamics imparted to biological systems by
  their evolutionary history. Statistical mechanics provides the key
  analytical approaches to abstracting the complex details of
  biodiversity into general macroscopic predictions that I show
  receive support from empirical data.  However, key deviations from
  the simplest statistical mechanics of biodiversity reveal the key
  role of biological evolution in driving systems away from the
  idealized steady state predicted by statistical mechanics.

  In Chapter 1 I expand a branch of non-equilibrial statistical
  mechanics, known as super statistics, to explain previously
  unaccounted for wild fluctuations in the richness of taxa through
  the Phanerozoic marine invertebrate fossil record and show how this
  non-equilibrium is driven by clades' punctuated exploration of their
  adaptive landscapes. This theory provides a novel explanation for
  deep time diversity dynamics invoking emergence of lineage-level
  traits as the drivers of complexity via the same mechanisms by which
  complexity emerges in large physical and social systems. In the
  context of fossil diversity I show how this complexity arises
  naturally from the uniquely biological mechanisms of punctuated
  adaptive radiation followed by long durations of niche conservatism,
  and thus identify these mechanisms as sufficient and necessary to
  produce observed patterns in the fossil record.  I test this theory
  using two seminal fossil datasets.

  In Chapter 2 I use the chronosequence afforded by the Hawaiian
  Islands to capture evolutionary snapshots of arthropod communities
  at different ages and stages of assembly to understand how the
  history underlying an assemblage determine its contemporary
  biodiversity patterns.  I apply static ecological theory of trophic
  networks based on statistical mechanics to these rapidly evolving
  ecosystems to highlight what about the evolutionary process drives
  communities away from statistical idealizations. This study
  indicates that rapid assembly from immigration and speciation in
  young ecosystems and extinction in old ecosystems could drive
  observed patterns.

  In Chapter 3 I highlight and explain the computational requirements
  to testing one statistical theory of biodiversity---the Maximum
  Entropy Theory of Ecology---with real data and make those test
  available in a stream-lined framework via the \texttt{R} package
  \texttt{meteR} that I authored.
\end{abstract}
